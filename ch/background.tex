\section{背景}

人类是具有社会性的,我们对协作(Collaboration)并不陌生。随着科技的发展,协作方式也日趋多样化。

\subsection{中心化协作的弊端}

中心化协作弊端:对上级负责,苹果等,依赖万能的一个人。这部分还没写完。

\subsection{去中心化协作的弊端}

原先去中心化协作的弊端:开源项目世纪难题,公地无人管理的问题。

智能资产引入去中心化协作之后的好处:不用当志愿者,自由意志得到发挥。

这部分还在写。

2008年10⽉31⽇,中本聪(Satoshi Nakamoto)提出了⽐特币的设计⽩皮书\footnote{https://zh.wikipedia.org/wiki/\%E6\%AF\%94\%E7\%89\%B9\%E5\%B8\%81},从此我们迎来了一个有区块链的世界。

区块链技术本质上是⼀种去中心化、⾮信任、基于博弈的⾃治体系,其真正的魅力是在去中⼼化思想下,基于共识机制的开放协作模式\footnote{https://nebulas.io/docs/NebulasWhitepaperZh.pdf}。
当前,社区⽤户对于区块链未来有⾮常大的期望,但区块链还远没有达到最终演化形态。区块链创新迎来空前的黄金窗口,与此同时也面临诸多挑战:

\begin{itemize}
\item \textbf{信息交互复杂度与日俱增:}

信息交互包括链内和链外,如“跨链”的数据和资产交互问题等。区块链正在从单⼀、简单的功能向复杂、多样发展。比特币是一种去中心化的电子加密货币,区块链可以记录交易信息;以以太坊为代表的第二代区块链提出了具有图灵完备性的智能合约,区块链变得可编程。随着区块链技术不断发展,信息交互复杂度与日俱增。当前不同的区块链系统之间相互隔离,形成“数据孤岛”,传统的链式结构无法满足所有应用场景的需求。

\item \textbf{硬分叉困境,系统升级举步维艰:}

为了应对信息交互复杂度提升,系统升级变得非常重要。而现有区块链的版本迭代往往引发“硬分叉”或“软分叉”,各类规则一旦早期确定,后期很难进行改变。协议进化缓慢,发展步伐受阻。如果选择“硬分叉”,又相当于割裂和分化社区,以太坊的“硬分叉”就导致ETH和ETC“双重资产”和社区分裂的“副作用”。

\item \textbf{用户角色越来越多,共识机制遭遇挑战:}

从早期比特币矿工、持币者,到开发者、应用使用者等,越来越多的人接触到区块链,利益分配受到挑战。工作量证明(Proof of Wook,PoW)仅专注挖矿激励,无法满足不同角色用户的利益需要;权益证明(Proof of Stake)及其变体摇摆在去中心化和中心化之间。去中心化的以太坊饱受诟病的升级缓慢受制于其共识机制,需要广泛认同才可执行升级,而整个生态中各方意见难以统一。另一方面,EOS干脆通过核心仲裁法庭对黑客进行判决,直接注销账户,无法反驳,又被认为是中心化的管理模式,部分利益集团的寡头共识。
\end{itemize}

目前不存在完善的解决方案可以解决上述问题。我们意识到,新技术的诞生众望所归。