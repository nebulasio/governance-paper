\section{星云治理基本原则}
\subsection{治理目标}
\begin{itemize}
\item 初级目标:实现开源软件社区的治理
\item 最高纲领:实现自治元网络
\end{itemize}
星云聚焦链上治理,致力于运用技术来提供更公平的环境,减小人为治理通常存在的摩擦和治理成本,用科技改变协作关系,促进社区良性发展,实现星云愿景。

\subsection{治理原则}
链上治理的最基本单位是一个星云主网地址。

每个地址有对应的私钥,私钥是对应地址控制权的唯一凭证。拥有以下基本权利,任何条件下不受侵犯:
\begin{enumerate}
	\item 拥有和操作星云链上资产的权利;
\item 发起提案的权利;
\item 参与提案投票的权利。
\end{enumerate}

\subsection{治理范围}
星云社区每一个人都享有使用星云并通过社区建设获益的权利。没有任何组织和个人干涉个人参与社区建设的自由。有如下两类情况需要社区集体决议:
\begin{enumerate}
	\item 该调整可能影响到星云链上所有人。如影响星云主网运行和使用的技术升级、参数调整等;或涉及到系统级别的风险管理,如系统遭受大规模黑客攻击等。
	\item 申请动用星云社区预留资金。包括主网研发、公共性质项目研发、社区拓展等。
\end{enumerate}

