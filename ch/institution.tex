\section{治理机构}
现阶段星云治理机构包括三部分:
\begin{enumerate}
	\item 星云理事会
	\item 星云基金会
	\item 星云技术委员会
\end{enumerate}
三个机构均允许社区所有人自愿报名加入。为了保障公正透明,有2个准则:
\begin{enumerate}
	\item 每个人最多同时加入两个机构。不可在任意时间段内同时在三个机构中担任职务。
	\item 三个机构互相制约,任何机构不具有独立决策和使用公共资金的权利。
\end{enumerate}
\subsection{星云理事会}
\subsubsection{权利和义务}
\textbf{星云理事会职能}

星云理事会职能仅限于保障和监督各项提案正常推进,推动星云愿景实现。不具备公共财产的处置权。

\textbf{理事权利}
\begin{enumerate}
	\item 提请提案”二次投票“的权利,7人共同商议决策,各自表决。
	\item 以星云理事会理事的身份进行推广。
	\item 委任星云技术委员会等组织机构或个人处理星云社区公共事务。
\end{enumerate}
\textbf{理事义务}
\begin{enumerate}
	\item 参与治理流程监督。
	\item 监督社区预留资金等公共财产安全。
	\item 宣传·推广星云理念。
\end{enumerate}
\subsubsection{组织结构}
\textbf{席位}

一共7名理事。

2019年第一届星云理事会理事,由星云基金会提名3席,社区公开选举产生4席。

星云基金会提名席位每两年至少减少一席。即最晚6年后星云基金会不再提名。

\textbf{任期}

2年/期,可连任1次

\subsubsection{选举}
\textbf{选举原则}

星云理事会理事通过链上治理流程投票产生。
\begin{enumerate}
	\item 只要拥有至少一个星云主网NAS账户地址的社区成员,都有选举与被选举权。
	\item 首期星云理事会选举方案由星云技术委员会代为提出。未来方案升级需经过链上公开投票通过方可执行。
	\item 社区成员对星云理事会有监督。
\end{enumerate}
\textbf{选举流程}

提名:公开阐述自己的贡献、治理理念和方案等。时长为一个自然月。

动员:候选人可在星云社区自行组织动员\&拜票。时长为一个自然月。

选举:公开链上投票。每人可选多人。时长不超过一周。

\textbf{获选条件}
\begin{enumerate}
	\item 需要获得所有投票参与者中,星云指数(Nebulas Rank,NR,核心星云指数考虑资产中值和出入度衡量,用于衡量一段时间内用户对整个经济体的贡献)不小于51\%的支持率,详见5.2投票机制。
	\item 取排名靠前者
\end{enumerate}
\textbf{中期投票和罢免}

任期满一年的理事需要做述职。社区根据述职情况进行中期投票,如果获得NR占比不小于51\%的支持率则继续任职,整个星云理事会需要重新选举。

如果遇到此种星云理事会集体罢免的情况,由星云技术委员会组织并监督新一轮星云理事会选举正常进行。在此期间,由述职通过的理事暂时代理星云理事会日常事务,但授权范围不扩大。

\subsubsection{资产监督}
\textbf{资产来源}

星云理事会不具有社区公共资产。仅负责监督社区公共资产的使用。社区公共资产主要包括2部分来源:
\begin{itemize}
	\item 35,000,000 NAS(35\%):社区预留

	(自星云创始起锁仓,解锁日期不早于2019年6月23日)
    \item 8,219.1744 NAS/日:共识记账系统增发
    
    (目前增发量为3\%,每日自动产生)
\end{itemize}
依照《星云非技术白皮书》(2017)所示,在2019年9月节点实现去中心化之后,共识记账系统增发将变为4\%,包括三部分:
\begin{itemize}
	\item 1\%:开发者激励协议原生激励(Developer Incentive Protocol,DIP),当前约为每周300 NAS
	\item 2\%:共识记账收入
	\item 1\%:理事会项目发展资金储备
\end{itemize}
在2019年9月节点实现去中心化之前,已经增发的部分由星云理事会负责管理,DIP部分暂由星云理事会管理。未来将要从中扣除已产生的服务器费用等相应记账开支(暂由星云基金会垫付,资产移交时进行扣除)。

\textbf{监督方案}

不同的资产用途有不同的监督方案。
\begin{enumerate}
	\item \textbf{项目协作平台}
	
	\textbf{方式:} 由星云社区全体参与成员公开审核,自动分发。
	
	社区自由提交资金申请,通过公开投票获得使用批准并且没有其他违规现象(参见2.2治理原则),即可申请到社区预留资金,在项目完成测试并上线后自动获得资金。
	
	\textbf{星云理事会职能:}如果某个项目存在违背治理基本准则的情况,星云理事会具有一次驳回重新投票再审的权利。
	
	\item \textbf{开发者激励协议原生激励}
	
	\textbf{方式:}自动,每周一次。
	
	星云是首个具有原生激励的公链。开发者激励协议(DIP)直接提供原生激励,每周一次自动向活跃DApp开发者的地址分发奖励NAS。
	
	\textbf{星云理事会职能:}链上自动化治理机制,星云理事会不具处置权,仅负责监督其安全性和稳定性。
	
	\item \textbf{星云基金会日常运作}
	
	包括星云基金会理事收入、紧急预案处理等无法计入项目制平台的事宜。此部分应有公示。
\end{enumerate}

\subsubsection{收益}
\textbf{总收入}

10,000 NAS/2年

\textbf{收益发放}

每满半年为一期,2年内分4次分发。每次分发额度为:3,000 NAS、4,000 NAS、6,000 NAS、7,000 NAS。如中期投票没有通过,后两期收入不予发放。

\textbf{财务要求}

为保证经济体利益的一致性,和星云理事会政策的延续性,理事正式入职时需要抵押100,000 NAS。卸任半年后解锁返还。

\subsubsection{公开透明}
星云理事会应确保治理流程和资产使用公开透明。
\begin{enumerate}
	\item 通过组织撰写季报等手段,定期公示资产和社区发展等情况。
	\item 如产生技术升级、项目申请驳回重新投票等处理情况,应及时公示。
	\item 所有人员选举情况、委任情况(如委任技术委员会管理项目协作平台)等,应及时公示。
\end{enumerate}

\subsection{星云基金会}
星云团队于2017年6月组建,星云基金会随之成立。负责星云团队的运维,保障项目正常进行。在实现《星云非技术白皮书》(2017)中承诺的技术点之前,星云基金会将继续肩负其历史使命。
\subsubsection{权利和义务}
\textbf{理事权利}
\begin{enumerate}
	\item 基金会内部选举和被选举权。
	\item 享受星云基金会运营带来的收益。
	\item 参与基金会发展和投资等的决策。
\end{enumerate}
\textbf{理事义务}
\begin{enumerate}
	\item 管理星云非公共预留部分的资产,包括星云团队成员奖励发放等。
	\item 按照星云发展需求组织生产,保障星云项目的研发可以正常进行,按时完成《星云非技术白皮书》(2017)承诺的开发内容和时间点。
	\item 每年一次对星云理事会述职,持续为星云生态服务。
\end{enumerate}
\subsubsection{组织结构}
\textbf{席位}

常务理事不小于5席,其中,主席1席,秘书长1席。

理事若干。

\textbf{任期}

1年/期,可连任

\subsubsection{选举}
\textbf{入选资格}

在星云基金会管理的团队资金池中,享有奖励额度达到80,000 NAS后即自动拥有进入星云基金会的资格。

\textbf{选举}

主席由现任星云基金会理事内部推选,支持率需要过半数。每位理事具有被选举权和选举权,一人一票。
由新上任的主席委任秘书长和其余常务理事,不少于3名。

\textbf{弃权}

每位具有加入星云基金会资格的人员,均有放弃理事资格的权利。
当常务理事不足5名时,则依照奖励额度排位候补。

\textbf{罢免}

星云基金会通过内部决议具有罢免星云基金会理事的权利。但结果必须向社区公示。
被罢免的星云基金会理事有权利向社区公开述职和要求通过全员投票重新裁决。

\subsubsection{资产管理}

\textbf{资产来源}

星云基金会管理的资产包括:
\begin{itemize}
	\item 20,000,000 NAS(20\%):星云团队预留
\item  5,000,000 NAS(5\%):星云社区发展基金剩余部分
\item 早期私募所得项目发展资金
\item 早期生态投资所得
\end{itemize}

\textbf{资产管理方案}

星云基金会资产发放采用多重签名的方案:5签3~5签5,则资金可用。(参考NAS当前流动性和总额进行加权,涉及使用的金额越大,需要的签名就越多。)

\subsubsection{收益}
\textbf{总收入}

\begin{enumerate}
	\item 星云基金会薪资
\item 工作每满1年,如述职通过,即可享受NAS奖励(由星云基金会常务理事负责分配)
\item 享受星云基金会生态投资等带来的收益
\end{enumerate}

\textbf{财务要求}

为保证经济体利益的一致性,和星云基金会政策的延续性,理事正式入职时需要抵押50,000 NAS。卸任半年后解锁返还。

\subsubsection{公开透明}
星云基金会应确保资产整体使用情况公开透明。
\begin{enumerate}
	\item 通过半年报等形式,披露整体经济情况和支出情况。
	\item 公示星云团队预留部分(20\%)的整体分发情况。
	\item 大于500,000 NAS的大额支出,需要向社区提前披露。 
\end{enumerate}

\subsection{星云技术委员会}
星云技术委员会成立于2018年9月。成立至今,星云技术委员会秉持着开放、共享、透明的精神,致力于推动星云链技术研发逐步去中心化、社区化。区块链技术为建设全新的、自激励的开源社区带来可能性。星云链的价值发现体系、自进化和自激励的技术理念为构建去中心化协作的世界提供了保障。星云技术委员会将全力推动星云愿景的实现。

自星云理事会成立后,原先由星云团队核心成员构成的星云技术委员会将完成历史使命,转型为社区化团队。星云技术委员会将受星云理事会委托,负责一系列社区化相关平台的日常运作,以及流程正当性监督。

\subsubsection{过渡期}
在2019年星云理事会第一次选举完成之前,星云技术委员会暂由星云团队成员代理。在原先星云技术委员会成员基础上进行拓展。星云理事会第一届理事人员确定后,星云技术委员会将启动改组。

\textbf{过渡期委员名单}

徐义吉、王冠、范学鹏、陈聪明、王卓尔、李晨、吴涤、温晨阳、鲁妍芬

\textbf{过渡期职责}
\begin{enumerate}
	\item 监督星云项目发展计划顺利进行
	\item 项目制平台和投票机制等社区化相关工具上线
	\item 第一次全网链上投票升级IR
	\item 监督星云理事会选举流程
\end{enumerate}

\subsubsection{权利和义务}
\textbf{委员权利}
\begin{enumerate}
	\item 参与项目审核和测试可以按照所审核项目价值提取项目申请额度的5\%作为奖励回馈,多劳多得。
	\item 享有星云技术专家团队的荣誉。
\end{enumerate}

\textbf{委员义务}

\begin{enumerate}
	\item 社区化提案质量监督。如果提案遭到举报,或者发现异常提案,三人核实后向理事会发起重审。
	\item 技术类提案的测试。技术把关。
\end{enumerate}

\subsubsection{组织结构}
\textbf{席位}

不限
\vskip 10pt

\textbf{任期}

1年/期,可连任

\subsubsection{选举}
自荐和社区推荐相结合,对社区公开述职。

候选人需通过技术委员会技术专家的技术能力考核,并通过全体成员投票(一人一票,支持率过半),方可决定是否加入。

\subsubsection{资产管理}
星云理事会委任技术委员会维护流程正当性,相关平台和工具正常运作。星云技术委员会本身不持有资产。

\subsubsection{收益}
\textbf{总收入}
\begin{itemize}
	\item 委托佣金(按月发放)
	\item 项目制相关审核和监督工作的5\%提成
\end{itemize}

\textbf{财务要求}

为保证经济体利益的一致性,和星云技术委员会政策的延续性,理事正式入职时需要抵押25,000 NAS。卸任3个月后解锁返还。

