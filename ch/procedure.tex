\section{链上协作和NAT投票激励}

星云链聚焦链上治理,致力于运用区块链技术来提供更公平的协作环境。

\subsection{链上治理流程}

星云治理通用流程如下:

\begin{enumerate}
\item \textbf{提案阶段}(Proposal Period):发起人在社区公开发起提案,在提案投票期间,通过了NAT链上投票;
\item \textbf{执行阶段}(Develop Period):项目立项,提案人本人或经本人认可的社区成员按计划执行;
\item \textbf{公测阶段}(Testing Period):执行人提交结果,在公测投票期间,通过了NAT链上投票;
\item \textbf{发布阶段}(NBRE Period):通过两次投票的项目,经技术委员会验收核实,且没有争议,即可最终执行发布。
\end{enumerate}

星云链技术开发和运营为项目制,项目的发起、立项、预决算、执行、审核、发布由星云技术委员会负责组织。星云技术委员会受星云理事会委任,只负责监督,保证整个流程公开透明。


\subsection{链上投票}
\subsubsection{基本原则}

星云链生态中的投票将通过星云主网实现,社区投出的每一票,将在星云链的区块上公开、透明的展现。在星云链的系统中,投票将遵循以下基本原则:
\begin{enumerate}
	\item 投票的最基本单位是一个星云主网地址
	\item 星云链的投票权重将参照地址的星云指数
	\item 对系统有积极贡献的行为应被奖励更多的投票权,在投票的场景中,我们认为投票行为是对星云系统有积极贡献的行为,应被激励更多的投票权
	\item 投票唯一的介质为星云指数的资产化表现,Nebulas Autonomous Token (NAT·星元币)
\end{enumerate}

\subsubsection{投票方式}

投票将通过星云链主网上的投票智能合约实现,每个地址可以选择投赞同、反对、弃权三种票。亦可不参与投票。

\subsubsection{投票的唯一介质:Nebulas Autonomous Token (NAT·星元币)}

\begin{enumerate}
	\item \textbf{名称}
		星元币 Nebulas Autonomous Token
		
	\item \textbf{代号}
	NAT

	\item  \textbf{形式}
	NRC20 token

	\item \textbf{发行量}
	NAT的发行量与星云链全网的星云指数(Nebulas Rank,NR)相关。在星云链全网星云指数恒定的情况下,NAT发行量恒定且释放的数量按周期递减,衰减系数为$\lambda$。
	
	NAT的初始发行总量将参照星云链主网Nebulas NOVA上线时的全网星云指数,总发行量为【1000】亿枚,初始设定$\lambda=0.99$, 在$\lambda$和全网NR指数不变的情况下,NAT的释放量在第二年衰减为前一年的60\%。 \xpcomment{(待确认—)} 

\xpcomment{(关于NAT的具体的发行的递减参数请参见【5.3】NAT算法)} 

	\item \textbf{发行方式}
	
	NAT将通过空投和质押星云链主网原生代币NAS两种方式,针对且只针对星云主网的用户发行。
	
	NAT的发行将分周期进行,定义每个发行周期为$i$,【且发行周期和投票周期一致】,NAT的发行将遵循如下原则:

		\begin{itemize}
			\item 每个空投周期内单位地址获得的NAT和该地址的NR正相关,下一个周期空投的比例在前个周期的基础上进行衰减,衰减系数为$\lambda$
			\item 地址在当前周期的投票行为会对当前周期空投的NAT数量有额外加成,加成系数为:$\mu$
			\item 用户质押星云主网的原生资产NAS达到6个周期以上,可以获得NAT
			\item 空投部分和质押部分不可同时获得
		\end{itemize}
\end{enumerate}

\textbf{空投部分}

在每个发行周期内,持有星云主网的地址,且在过去6个空投周期内的NR加权不为零的地址,可以获得NAT空投。

在单一的发行周期内将进行两次空投,
1. 在空投周期初始,空投比例将参照以下函数,使得空投的比例和用户地址的NR值成正比,且空投比例会随着时间进行衰减。在越靠前的周期内,同一地址获得的空投的比例越高。
\xpcomment{(插入A部分的函数)} 

2. 在空投周期结束前,按照用户在此周期内的投票行为进行第二次空投,此次空投将给予参与投票的用户额外的NAT加成。用户投出赞成、反对和弃权票均可获得加成。如此次空投周期中,用户未参与投票,则无法获得此部分加成。加成部分按照一下函数计算:
\xpcomment{(插入B部分的函数)} 

\textbf{质押部分}

用户向投票的智能合约质押NAS可以获得一定比例的NAT返还,用户将按照自己质押NAS占全网总质押NAS的比例瓜分质押部分的NAT,在每个发行周期内,质押部分返还的NAT将按照一下函数计算:
\xpcomment{(插入C部分的函数)} 

\textbf{NAT的使用场景}

NAT将作为星云生态中投票场景的唯一介质,支持星云推进社区的治理。社区成员可以通过使用NAT投票表达自己自己对星云生态的意见,包括但不限于星云理事会的选举、星云主网NBRE中执行的IR及星云项目提案的立项表决等。

\textbf{管理}

用户可以在NAS Nano Pro及其他支持NRC20的钱包中管理自己的NAT。同时,用户可以在支持星云链的区块链浏览器上(比如explorer.nebulas.io)查看NAT的交易、流通情况等。

\textbf{NAT的黑名单地址}
目前,考虑到中心化交易所的地址上的资产为代为用户管理的资产,在当前的技术条件下,无法判断投票行为是否可以真实反映用户的意愿。因此目前,在星云生态的投票场景中,中心化交易所的地址将作为黑名单地址处理,不具备投票权亦不能享受投票行为的加成部分。但是中心化交易所的地址作为星云主网的地址仍具有和其他地址一样的使用和管理资产的权利,仍可以获得NAT的空投。

\textbf{投入投票智能合约的NAT}在每个发行周期内,用户投入星云链投票智能合约的NAT将被立即烧毁,烧毁的比例会按照周期递减,递减速率和NAT发行的递减速率一致。在每个周期内,未被烧毁的NAT,将会在每个周期的第二次空投的过程中返还给本周期内参与投票的用户。每个周期内烧毁部分的NAT将按照一下函数计算:
\xpcomment{(插入NAT烧毁速率的函数)}

\textbf{投票的手续费}
每期投出的NAT中的1\%将作为投票手续费,此部分手续费为星云生态中的公共资产。由社区共同监管,并由星云理事会监督此部分资产使用的正当性。星云理事会需要将此部分资产公示并定期向社区公示资产的使用情况。星云理事会不具有手续费的所有权,亦不能将手续费直接用于投票。

\subsubsection{投票规则}

\begin{itemize}
	\item \textbf{投票的监督}:公开投票由技术委员会负责组织和管理。公开投票接受所有人的公开监督。针对违反社区治理基本准则的提案,技术委员会可以向理事会发起重审提案申请。星云理事会作为星云社区治理的监管机构,有权对任何提案的投票发起二次投票。对于同一提案,理事会可以且仅可以发起一次二次投票。
	\item \textbf{项目投票通过的标准}
	
	投票是否通过将通过两个维度的标准来衡量: 1) 投票的参与度 2)赞成票的占比

	1) 投票的参与度:

	对于涉及到使用公共资产支持的项目,投票的参与度不得低于该项目提起的资产的金额占全网的的比例。
	如某提案要求动用X个NAS支持,此时星云主网中流通的NAS为Y。(此处定义的流通的NAS为任何未在锁仓/质押状态的,可随时在星云主网上进行转账交易的NAS)
	则此项目通过需要达成的全网的投票参与度不得低于X/Y,换算成NAT来表示,及参与到此次项目中投票的NAT与该周期初期空投给用户的NAT的比例不得低于X/Y

	对于不涉及到使用公共资产支持的项目,投票的参与度不得低于51\%,此类项目包括但不限于星云主网参数的调整、NBRE要执行的NPR等

	2)赞成票的占比
	在满足投票最低参与度之外,某一项目投票通过还需要满足赞成票占总投入票数的比例不得低于51\%
	即假设某一项目共收到票数N,其中赞成票数为Y,反对票数为N,弃权票数为A,则只有当Y/(Y+N+A) >= 51\%时,此项目才被视为投票通过

	\item \textbf{二次投票}
	
	根据前文所述,星云理事会会作为星云生态中治理流程正当性的监督者,有权且仅有权对某一项目提起一次“二次投票”的要求。当理事会发起“二次投票”的要求时,该项目被视为进入到新的投票周期进行一次新的投票。第一次投票过程中的NAT不予返还,会按照当周期的烧毁速率进行烧毁。
	在二次投票的过程中,该项目投票的参与度需大于第一次投票的参与度。即假设第一次投票的参与度为X/Y,则第二次投票的参与度应大于X/Y,且赞成票的比例不低于51\%方可视为投票通过。
	 
\end{itemize}




