\section{治理流程}
在NBRE中,升级通过提交IR(LLVM IR)来进行。不同IR的提交会对星云链整体系统环境的性能产生实质性的影响,IR的提交实际程度上决定了星云链主网的升级和性能推进方向。

现阶段星云功能特点未全部实现,星云团队会慎重开放提交IR的权限。当前治理流程如下:

\textbf{治理流程(现阶段)}	
\begin{enumerate}
\item \textbf{提案阶段}(Proposal Period):在社区公开讨论提案,发起项目,并可申请使用社区预留资金;
\item \textbf{投票阶段}(Proposal Voting Period):投票区块期间,赞同票NR占比超过全网NR值一定比例x\%,视作通过;
\item \textbf{开发阶段}(Develop Period):提案人进行开发,或在社区寻找开发者共同开发,开发过程在协作平台(go.nebulas.io)上公开;
\item \textbf{测试阶段}(Testing Period):经主网团队操作,部署测试网公测,关于IR是否可以部署到主网进行链上加权投票,投票区块期间,赞同票NR占比超过全网NR值不小于x\%,视作通过;
\item \textbf{发布阶段}(NBRE Period):经技术委员会验收核实,且测试期内没有争议,IR被NBRE执行,并入主网。
\end{enumerate}

\textbf{治理流程(NF实现后)}

NF实现后,任何人可以提交IR,以主网NF侧链形式存在,不需要经过主网团队审批即可自行测试。

\subsection{提案}
星云链技术开发和运营为项目制,项目的发起、立项、预决算、执行、审核由星云技术委员会负责组织。星云技术委员会只负责监督,保证整个流程公开透明。如涉及线下活动,应符合当地相关法律法规。原则上星云技术委员会不介入其中,并且不承担任何责任。

\textbf{提案人}

所有社区成员。

\textbf{提案方式}

在提案专区公开发出提案。

\textbf{提案内容}

一份完整的提案需要包括以下内容:
\begin{enumerate}
\item 提案人姓名(需要登录)
\item 提案描述(当前问题、解决方案、理由)
\item 提案成果形式(如github地址、DApp、PDF、文稿链接等,也可以是综合性结果或报告)
\item 指定执行人,无指定执行人则可以在社区公开招募执行人
\item 各时间点设定:投票开始和结束时间、招募执行人时间、执行阶段时间点和截止日期、执行耗时预估、测试时长
\item 如果申请使用社区预留资金,需要填写明确金额(NAS作为计价单位)
\end{enumerate}

\textbf{提案时长}

由提案人自由定义。

\textbf{提案修改}

在投票正式开始之前,提案人可以修改提案所有内容。其他人可以评论、关注等。每次修改需要支付$n^2$ NAS。这部分奖励会均分给当次提案前参与过讨论的用户。

\textbf{提案取消}

在正式投票开始之前,提案人可以取消提案。正式投票开始后,不可取消提案。

\subsection{投票}
\subsubsection{星云自治币}
星云投票使用星云自治币(Nebulas Autonomous Token,NAT)进行。

\textbf{名称}

星云自治币 Nebulas Autonomous Token

\textbf{代号}

NAT

\textbf{形式}

NRC20 token

\textbf{发行方}

星云理事会

\textbf{作用}

用于星云治理。

\textbf{发行方式}

NAT是星云指数(Nebulas Rank,NR)的累计值。

按照《星云指数黄皮书》(2018年6月30日)所述,星云指数用以衡量区块链世界链上数据价值,以流动性、传播性、互操作性等体现数据互动关系的因素为基础,可以用来衡量地址、智能合约、去中心化应用等对象的影响力。其中,核心星云指数(Core Nebulas Rank)考虑资产中值和出入度衡量,用于衡量一段时间内该对象对整个经济体的贡献。星云指数是反应贡献度的良好指标,也是星云独创技术能力之一。基于此,NAT作为NR的累计值,可以反应一段时间内该地址持有者的贡献度。

NAT按照一定的算法,基于NR等比折算而来,例如某日NR为$X_1$,则NAT加$f(X_1)$,次日NR变化为$X_2$,则NAT增加$f(X_2)$。某日NR=0,则NAT当日不增加。最后NAT为$f(X_1)+f(X_2)+\cdots$。NR越高,获得的NAT也越高。即,用户可以通过提高活跃度、提高持币量等方式提高自己的NR,从而自动获得更多NAT。具体参见算法。

\textbf{发行量}
	
NAT的发放按比例衰减。NR是地址对经济体作出贡献的相对比例,NAT情况类似,不会产生无限超发的情况,发行量可控。
但NAS的4\%的共识记账增发同时会反映在NAT的分发上。

\textbf{交易}

任何人可以进行NAT交易。NAT为正常NRC20 token。
NAT是否上交易所,不禁止也不提倡,由市场决定。

\textbf{管理}

用户可以在Explorer上(比如explorer.nebulas.io)像查看其他NRC20 token一样查看NAT的交易、流通情况等。
用户可以在进行链上投票治理的提案平台看到自己的累计NAT数量,并提取出来,执行投票等治理动作。

\textbf{锁仓}

另有专项NAS锁仓和长期投资项目。NAT本身没有官方锁仓返息计划。

\textbf{使用NAT进行投票的理由}

星云指数具有三大特点:
\begin{itemize}
	\item 真实:衡量对经济系统的贡献度
\item 公平:有效抵抗操纵的计算函数
\item 多样:根据不同场景参数可扩展
\end{itemize}
基于此,NAT不仅可以反应对经济体的贡献度,并且天生具有抗作弊能力、自适应能力。可以应对投票过程中可能产生的舞弊行为。

另外,使用NAT进行投票,避免了用户直接使用大额NAS进行转账的安全隐患。大大简化投票难度,提高安全性。

\subsubsection{投票方式}
\textbf{投票种类}

每人可投:赞同、反对两种票。亦可不参与投票

\textbf{投票时长}

提案人自定义。但必须限定投票时长,不能无限投票。技术委员会可以给出指导建议。

\textbf{投票管理}

公开投票由技术委员会负责组织和管理。公开投票接受所有人的公开监督。针对违反社区治理基本准则的提案,技术委员会可以向理事会发起重审提案申请。

\textbf{投票流程}
\begin{enumerate}

\item 用户投票

用户在提案平台可以查看。每个地址针对一个提案可以投X NAT

\item 烧毁NAT

投出的NAT立即全额烧掉。

\item 投票结束后返还NAT

投票截止时间到了之后,把投票期间NR增加的部分加上烧掉的NAT部分,以NAT形式按NR比例返回到参与投票的地址。
\end{enumerate}

\textbf{投票结果}

满足以下条件算投票通过:

\begin{enumerate}
\item 赞同票总占比大于51\%
\item 投票期间,参与投票的NAT占比超过全网NAT值一定比例x\%,x\%的决定因素为提案申请的金额占链上流通量比例。
\item 如果金额为0,x\%有保底值。
\end{enumerate}

\subsubsection{投票收益}
\begin{enumerate}
	\item 参与投票则有机会享有经济体活跃度提升带来的红利。
	\item 参与投票有机会参与项目经济体流转。
	\item 没有参与投票的地址也有小额返利。
\end{enumerate}

\subsubsection{公开透明}
投票过程时刻保持公开透明可见。投票结果不可修改。并且通过技术手段保障流程正当性。

\textbf{技术合法性}
\begin{enumerate}
\item 链上投票保证了投票过程的公正性和安全性。同时若引入匿名制还能有效保护成员隐私(隐藏其投票内容)。
\item NR的抗作弊性质保证了投票过程的公平性:成员建立大量新账户不会提高他的投票效用,同时高资产成员将其资产分散到多个子账户亦不会提升其投票总效用。
\item 投票所采用的各项技术是与Nebulas NOVA的技术一脉相承的,如DIP本质上实现的就是用户给DApp投票的过程。同时我们今后也会不断对投票技术进行补充和完善,将其作为一个不可分割的整体。
\end{enumerate}

\subsection{NAT算法}
\textbf{空投部分}

如前所述,每个投票阶段(一般而言为两周)根据每个地址的NR值获取增发的NAT。若NR值为$x_1$,则增发的NAT为$\lambda f(x_1)$

其中$\lambda \leq 1$为衰减比例参数,随当前市场现存总NAT而变化(总NAT越多$\lambda$越小)。$f(\cdot)$为NR到NAT的映射函数,递增,且满足$f(0)=0$。

现阶段$f(x)=x$是一可选方案。



\textbf{投票部分}

每个用户支付的用于投票的NAT立即销毁。

\textbf{返还部分}

投票过程完成后,为了激励正常投票的用户,投票过程结束后会根据用户所投票数返还一定比例的NAT。具体数额为
$$\mu C \min \{\frac{C_0}{C},1\}$$
其中$\mu>1$为激励系数,可选取$\mu=1.1$。$C$为用户此阶段(一个增发周期内)投票花费(即烧毁的)NAT总数目,$C_0$为预期希望用户投出的NAT数目,每个用户对应不同的$C_0$,一般而言选取$C_0=\lambda f(x_1)$,即我们希望用户把此阶段增发的NAT用于投票,而不推荐用户一次投出过多的票。

对于用户而言,假设他一个阶段内用$C$枚NAT进行投票且$C \leq C_0$,则其NAT的变化量为
$$\mu C \min \{\frac{C_0}{C},1\}-C = \mu C_0-C$$
上式随$C$递减。这意味着对于一个利益最大化的用户而言他的最优策略为投出$C_0$票。如若项目对其意义重大,需要投出大量NAT,则有可能$\mu C_0-C <0$,相当于该用户要“动老本”进行投票。

若$C < C_0$,则其NAT的变化量为

$$\mu C \min \{\frac{C_0}{C},1\}-C = (\mu-1)C$$
上式随$C$递增,这意味着最优策略同样是投出$C_0$票。当投票数目较少时仍能获得收益但相对投$C_0$而言较少。

[通过后,此处查变化曲线图]

注意:
\begin{itemize}
\item 目前版本暂定赞同票与反对票没有区别,即返还比例不同。之后可根据票种设定并乘上不同的返回参数$\mu_1$
\item 若考虑到投票完成后系统的总NR变化,则可再乘上一个系数$\mu_2$用于反应该项目对系统繁荣度的贡献。
\item 若上述两条采用最终返回NAT数值为
$$ \mu\mu_1\mu_2 C \min \{\frac{C_0}{C},1\}$$


\end{itemize}

\textbf{性质}

上述投票算法具有下列良好性质。
\begin{enumerate}
	\item 抗滚雪球效应:如若简单的按固定比例返还NAT,则一个用户可以每次投出所有的NAT并享受大于1比例的返还(如1.1),则其总NAT将按$1.1^n$指数级上升,增长过于庞大。
	\item 抗收买性:若一个低NR的用户以购买的方式获取大量NAT并用于投票,由于对低NR用户我们设定的对应$C_0$较低,返还的NAT很少,大部分都被烧毁,导致该用户剩余NAT很少作为惩罚。
	\item 抗通货膨胀:由于系统增发NAT比例与当前市场NAT总量有关,可有效控制NAT的贬值。
\end{enumerate}

\subsection{执行}
\textbf{提案人义务}
\begin{enumerate}
\item 监督执行人执行,各时间点没有错失。
\item 维护社区讨论。
\end{enumerate}

\textbf{提案人权利}
\begin{enumerate}
\item 享有提案中涉及的应得回报。
\item 在申请参与执行的社区用户中挑选合适的人来执行。
\end{enumerate}

\textbf{执行人义务}
\begin{enumerate}
\item 定期公示进度,保证执行过程公开透明。
\item 保证按时保质完成,并按照提案要求提交成果。
\end{enumerate}

\textbf{执行人权利}
\begin{enumerate}
\item 享有提案中涉及的应得回报。
\item 在不影响进度并且能达成最终目标的情况下,有权调整具体的执行步骤。
\end{enumerate}

\textbf{执行时长}

考虑到币价波动,建议不超过一季度。复杂度高的项目建议拆分阶段进行。

\textbf{提前终止}

如果开发没有按时完成,提案立即提前宣告失败。社区预留资金不会发放。

\subsection{公测}
\textbf{公测步骤}

完整的公测需要经过以下3个步骤:
\begin{enumerate}
\item 提交项目成果。开始公开测试。
\item 投票。二次投票方法和时长与一次投票相同,NAT占比y\%应大于等于x\%。
\item 技术委员会审核。通过二次投票的技术类提案,会随机派给三名技术委员会相关技术人员进行复审,全部通过即视为通过,进入发布流程。
\end{enumerate}

\textbf{测试时长}

技术类项目按照开发时长折算,最短时间为开发周期的10\%。

\subsection{发布}
非技术项目通过公测后即视为成功。技术项目需要进行链上数据更新。

\textbf{链上数据更新流程}

当协议需要升级之时,包含升级的代码会被包含在一个特殊的交易当中。这类交易只能由星云理事会暂时代为控制的主节点发起,并广播到全网。之后所有节点一旦接收到这类交易,NBRE将会自动执行其中包含的升级代码,完成更新。与此同时,核心协议的旧版本仍会在链上,以作参考。注意,节点拒绝更新只能重写本地代码。

\textbf{技术相对优势}
\begin{enumerate}
	\item 其他区块链项目升级主要还是通过硬/软分叉的方式,星云升级不需要进行分叉。
\item 交易所等不需要充停。
\item 高效率,零成本。
\end{enumerate}

举例:每个区块是火车上的一节车厢,那么每个区块中的交易就相当于车厢中的乘客。之前,如果火车发生故障,我们需要把它送到工厂维修,就像是我们手动修复漏洞,升级代码。当主网发现漏洞之后,这时,火车不能继续运行,车上的乘客也会因此受到影响。而NBRE的升级方式相当于把零部件提前放在火车上,用一节车厢专门储存零部件,当火车发生故障之时,车上的工人就可以及时换上新的部件。在此过程中,火车仍可正常运作。

\textbf{奖励发放}

发布主网后自动发放所有相关人员的奖励。

