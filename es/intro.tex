% !TEX root = main.tex

\section{Introducción}

\textbf{Palabras clave: Gobernanza\ Descentralizado\ Colaborativo\ Incentivo\ Auto-evolutivo\ Autónomo}

\vspace{2em}

Nebulas es un \blockchain público, de código abierto, enfocado en la creación de una
\textbf{metanet autónoma}~\cite{AutonomousMetanet} cuya meta es la de utilizar datos \onchain para las interacciones y la colaboración entre usuarios. Nuestro lema es \textbf{que todos obtengan valor de la colaboración descentralizada, de una forma justa, por medio del uso de la tecnología \blockchain.}~\cite{vision}

Este \textit{Libro Naranja} explicará la forma en que Nebulas utiliza su innovadora tecnología para crear un \textbf{modelo colaborativo que, con la ayuda de tecnologías innovadoras únicas, permitirá administrar activos públicos \onchain y crear la \DAO~\cite{DAO}, que le brindará incentivos y capacidad auto-evolutiva al sistema}. La gobernanza de Nebulas se ha diseñado teniendo en cuenta los siguientes tres aspectos, que se explicarán en detalle más adelante en este documento:

\begin{enumerate}
	\item \textbf{Organización y supervisión}:
	Los Grupos de Comunidad de Nebulas (\textit{Nebulas Community Groups}) operarán de forma independiente y se controlarán entre sí. Se crearán, además, los siguientes órganos: el Concejo de Nebulas (\textit{Nebulas Council}), la Fundación Nebulas (\textit{Nebulas Foundation}), el Comité Técnico de Nebulas (\textit{Nebulas Technical Committee}). Para todos ellos se detallará su composición, poderes y obligaciones.
	\item \textbf{Colaboración \onchain}:
	Es el proceso de colaboración comunitaria y la actualización del sistema mediante la emisión de votos NAT \onchain (\ref{nat}).
	\item \textbf{Economía e incentivos}:
	La descripción de la economía de los votos NAT \onchain, el proceso de gobernanza de Nebulas y la forma en la que este modelo económico provee incentivos a cada miembro de la comunidad.
\end{enumerate}