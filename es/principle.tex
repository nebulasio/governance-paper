\section{Descripción general}

\textbf{La clave central de la gobernanza de Nebulas es que su ejercicio se realice \onchain; su principal objetivo son los activos \onchain, y el método básico de coordinación es la interacción \onchain}; esto apunta a resolver el dilema de gobernanza y a diseñar un mejor modelo de colaboración descentralizada que permita hacer realidad la visión de Nebulas.

\subsection{Tecnología}

El proyecto Nebulas —de código abierto y basado en \blockchain público— fue diseñado en 2017 con el fin de brindar el marco técnico para su gobernanza (y para materializar su visión). Se puede describir como una metanet autónoma que utiliza metadatos híper-mapeados para resolver problemas complejos de datos e interacciones, y cuya habilidad central es la de valuar los datos \onchain mediante \nr. Además, hace uso de un nuevo mecanismo de consenso y de su capacidad de auto-actualización para resolver problemas complejos de colaboración; ofrece incentivos duraderos a sus usuarios (por medio de \textit{Nebulas Incentive, NI}) y posee la capacidad de actualizarse sin necesidad de \forks (a través de \nforce).

Nebulas hace uso de la tecnología para reducir la fricción y los costos asociados a las gobernanzas humanas, para cambiar las relaciones colaborativas y para promover el desarrollo de comunidades saludables. Para más detalles acerca de la tecnología de Nebulas, refiérase al \textit{Libro Blanco Técnico}~\cite{TechWhitepaper}.

\subsection{Tres derechos básicos}
\label{rights}

All complex systems begin with the development of basic rules that follow logical steps. Since the basic component of blockchain assets is an \textbf{address} this is also the basic unit of the Nebulas community governance. Therefore, we formally propose three basic rights for each Nebulas address:

\begin{enumerate}
	\item The right to own and utilize assets on Nebulas.
	\item The right to initiate a proposal.
	\item The right to vote.
\end{enumerate}

Nebulas strongly believes that each address has the above listed fundamental rights within the system and will not be infringed upon under any circumstance. Any user who has access to a Nebulas address via their unique private key has the \textbf{right} to control their assets. Without any absolute centralized organization or individuals, each and every member of the Nebulas community has the freedom to use the mainnet and participate in the decision making process. Members can also participate in the production and construction of community approved projects.

Nebulas governance is based on these three rights. Simply put, anyone can create a proposal, share it with the community, and ultimately, have the community approve their proposal via an on-chain voting system. This means that the future of Nebulas is in the hands of every participating community member!

\subsection{Governance scope}


The public assets primarily controlled by Nebulas governance include:

\begin{enumerate}
	\item Intellectual property, including public, open source code (such as Nebulas Mainnet upgrades and other related codes that affect the public interest of Nebulas).
	\item Community public assets according to the \textit{Nebulas Non-technical Whitepaper} (~\ref{supervision}).
\end{enumerate}

In general, blockchain is a network that tracks collaboration relationships as well as a network that tracks \textbf{assets} of incentive cooperation. In a system that is absent of any centralized power, public assets should be managed by all community members.

At the same time, the scope of Nebulas governance is limited to Nebulas public assets and Nebulas governance provides the basic governance tools for the Nebulas community. Organizations within the Nebulas community (such as DApp project parties, exchanges, etc.) can use the Nebulas governance tools (such as NAT on-chain voting) to promote the ecological development of their projects; however, the Nebulas Community Group Will not \textbf{take up} the role of judge. With off-chain events, Nebulas community members should comply with local laws and regulations. The previous chapter (~\ref{background}) described different situations that should adopt a matching governance model; Nebulas governance will not violate the original intention of the design to blindly expand it's governance range.

\subsection{Features}

There are three primary features of Nebulas governance:

\begin{enumerate}
	\item

	\textbf{The same rules for all, open and transparent.}

	Everyone co-exists and develops under standardized rules. At the same time, any new requirements are defined by the initial base rules.

	\item

\textbf{The decentralized collaboration of a prospering economy.}


	\begin{enumerate}
		\item

		\textbf{Decentralize the process of community collaboration:} On-chain governance is the core of Nebulas governance and allows for community oversight of the process.

		\item

		\textbf{Decentralize the governance of public assets:} As a decentralized community with asset attributes:

		\begin{itemize}
			\item The Nebulas Community Groups will ensure the legitimacy of the governance process and the mutual restriction of power; no organization or individual has absolute power and no organization or individual can directly use public assets.
			\item Provides technical support for asset governance and security through the original Proof of Devotion (PoD) consensus mechanism.
		\end{itemize}

	\end{enumerate}

	\item

	\textbf{Incentivized community for a high participation rates.}

	Lasting positive incentives are the core of community organizations and the cornerstone of autonomy.

	The Core Nebulas Rank can be combined with a variety of parameters to determine the contribution of an address within the entire ecosystem ~\cite{yellowpaper}. Based on this, not only miners and currency users, but also developers, active users and others in different roles can be a source of relatively regular, quantitative contribution to the entire ecosystem. All users can also be compared with one another and in return, Nebulas can inspire everyone in the ecosystem according to their contributions.

	Moreover, by utilizing the \textbf{asset-based} Nebulas Rank and actively participating in on-chain governance (such as on-chain voting), users can receive NAT incentives by contributing to the community and ecosystem. The NAT token is the native incentive of the Nebulas Rank algorithm which is implemented through technical capabilities rather than personal intervention which reduce the likelihood of individual manipulation on the network.

	Three basic perceptions about ecosystem motivation:

	\begin{enumerate}
		\item Positive incentives are the basis for ensuring benefits for everyone. Incorrect or uneven distribution of incentives can lead to bad money driving out good money.
		\item Incentives should be continuous; short-lived incentives can cause irreversible, negative results.
		\item The scale of incentives should be appropriate.
	\end{enumerate}

	Nebulas always regards incentives as an essential part in designing the technical features of the Nebulas economy. Positive incentives are expected to benefit community members more equitably and significantly increase community engagement.

	\item

	\textbf{Inclusive and efficient collaboration.}

	Since Nebulas is a true autonomous metanet, there is capacity to achieve self-evolving without hard forks. Within the Nebulas community, once a proposal is approved via on-chain voting, an upgrade can be completed and iterated immediately. If a problem arises on the network, improvements can be quickly be released to the entire network. Future problems on Nebulas will not be like those on existing public chains such as Ethereum which is bound by its immature technologies and strategies.

	While technically efficient, Nebulas governance also offers a transparent and straightforward process (\ref{governance}) to improve collaboration efficiency.

\end{enumerate}