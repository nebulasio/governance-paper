\section{Descripción general}

\textbf{La clave central de la gobernanza de Nebulas es que su ejercicio se realice \onchain; su principal objetivo son los activos \onchain, y el método básico de coordinación es la interacción \onchain}; esto apunta a resolver el dilema de gobernanza y a diseñar un mejor modelo de colaboración descentralizada que permita hacer realidad la visión de Nebulas.

\subsection{Tecnología}

El proyecto Nebulas —de código abierto y basado en \blockchain público— fue diseñado en 2017 con el fin de brindar el marco técnico para su gobernanza (y para materializar su visión). Se puede describir como una metanet autónoma que utiliza metadatos híper-mapeados para resolver problemas complejos de datos e interacciones, y cuya habilidad central es la de valuar los datos \onchain mediante \nr. Además, hace uso de un nuevo mecanismo de consenso y de su capacidad de auto-actualización para resolver problemas complejos de colaboración; ofrece incentivos duraderos a sus usuarios (por medio de \textit{Nebulas Incentive, NI}) y posee la capacidad de actualizarse sin necesidad de \forks (a través de \nforce).

Nebulas hace uso de la tecnología para reducir la fricción y los costos asociados a las gobernanzas humanas, para cambiar las relaciones colaborativas y para promover el desarrollo de comunidades saludables. Para más detalles acerca de la tecnología de Nebulas, refiérase al \textit{Libro Blanco Técnico}~\cite{TechWhitepaper}.

\subsection{Tres derechos básicos}
\label{rights}

Todos sistema complejo nace a partir del desarrollo de un conjunto de reglas claras que a su vez siguen distintos pasos lógicos. Dado que el componente básico de un activo \blockchain es una \textbf{dirección}, esta es también la unidad fundamental de la gobernanza de la comunidad de Nebulas. De ese modo, proponemos formalmente tres derechos básicos para cada dirección Nebulas:

\begin{enumerate}
	\item El derecho a poseer y utilizar activos en Nebulas.
	\item El derecho a iniciar una propuesta.
	\item El derecho a votar.
\end{enumerate}

Nebulas cree firmemente que cada dirección posee, dentro del sistema, los derechos fundamentales arriba mencionados, y que no deben ser vulnerados bajo ninguna circunstancia. Todo usuario con acceso a una dirección Nebulas mediante su clave privada única tiene el \textbf{derecho} a controlar sus activos. Gracias a la ausencia de una organización (o un individuo) centralizada y con poder absoluto, todos y cada uno de los miembros de la comunidad de Nebulas tienen la libertad de usar la red principal y participar en el proceso de toma de decisiones. Los miembros también pueden participar en la producción y la construcción de proyectos aprobados por la comunidad.

La gobernanza de Nebulas se basa en esos tres derechos. Para decirlo de un modo sencillo: cualquier persona puede crear una propuesta, compartirla con la comunidad y, en última instancia, lograr la aprobación de dicha comunidad a la propuesta mediante un sistema de votación \onchain. Esto significa que el futuro de Nebulas está en manos de cada participante miembro de la comunidad.

\subsection{Alcance de la gobernanza}

Los activos públicos controlados principalmente por la gobernanza de Nebulas incluyen:

\begin{enumerate}
	\item Propiedad intelectual, incluyendo código fuente abierto y público (tales como las actualizaciones de la mainnet de Nebulas, y otro código relacionado que afecta el interés público de Nebulas).
	\item Activos públicos comunitarios de acuerdo al \ntechw (~\ref{supervision}).
\end{enumerate}

En general, un \blockchain rastrea las relaciones de colaboración de su red, así como también rastrea los \textbf{activos} otorgados como incentivos de cooperación. En un sistema en el cual no existe un poder centralizado, los activos de carácter público deben ser administrados por todos los miembros de la comunidad.

En este sentido, la gobernanza de Nebulas está limitada a sus activos públicos, y provee herramientas básicas para su comunidad. Las organizaciones establecidas dentro de la comunidad de Nebulas (tales como \dapps, casas de cambio, etc.) pueden hacer uso de las mencionadas herramientas (por ejemplo, el sistema de votación \onchain por medio del token NAT) para promover el desarrollo ecológico de sus proyectos; no obstante, el Grupo Comunitario de Nebulas no \textbf{tomará} el papel de juez. En el caso de eventos \textit{off-chain}, los miembros de la comunidad de Nebulas deberán cumplir con las leyes y regulaciones locales.

El capítulo anterior (~\ref{background}) describe diferentes situaciones en las cuales se debe adoptar un modelo de gobernanza acorde; en el caso de la gobernanza de Nebulas, no se violará la intención original de su diseño con el fin de expandir ciegamente el alcance de aquella.

\subsection{Características}

Existen tres características fundamentales en la gobernanza de Nebulas:

\begin{enumerate}
	\item

		\textbf{Las mismas reglas abiertas y transparentes para todos}

		Todos coexisten y desarrollan bajo reglas estandarizadas. Al mismo tiempo, todo nuevo requerimiento será definido por las reglas básicas iniciales.

	\item

		\textbf{La colaboración descentralizada en una economía próspera}

		\begin{enumerate}
			\item

			\textbf{Descentralizar el proceso de colaboración comunitaria:} la gobernanza \onchain es la piedra angular del futuro de Nebulas, y es lo que permite que la comunidad supervise el proceso de expansión.

			\item

			\textbf{Descentralize la gobernanza de los activos públicos:} como comunidad descentralizada con atributos de activos,

			\begin{itemize}
				\item los grupos comunitarios de Nebulas garantizarán la legitimidad del proceso de gobernanza y la restricción mutua de poder; ninguna organización o individuo tendrá poder absoluto, y ninguna organización o individuo puede utilizar directamente los activos públicos.
				\item provee soporte técnico para la gobernanza de los activos y su seguridad, a través del mecanismo de consenso original \pofd.
			\end{itemize}

		\end{enumerate}

	\item

	\textbf{Incentivar a la comunidad a una alta tasa de participación.}

	Los incentivos positivos duraderos son el núcleo de las organizaciones comunitarias y la piedra angular de la autonomía.

	Los \cnrank se pueden combinar con una variedad de parámetros con el fin de determinar la contribución de una dirección dada al ecosistema en su conjunto ~\cite{yellowpaper}. Basándonos en esto, no sólo los mineros y los usuarios de los activos, sino también los desarrolladores, los usuarios activos y otros miembros en diferentes roles, pueden ser fuentes de contribuciones relativamente regulares y cuantitativas al ecosistema. También es posible comparar distintos usuarios entre sí y, a cambio, inspirar al ecosistema entero de acuerdo a sus contribuciones.

	Más aún, al utilizar el sistema \nr —\textbf{basado en activos}— y participar activamente de la gobernanza \onchain (al votar, por ejemplo), los usuarios pueden recibir incentivos NAT.

	El token NAT es el incentivo propio del algoritmo Nebulas Rank, que está implementado por medio de capacidades técnicas más que intervención humana, lo que reduce las chances de manipulaciones en la red.

	Existen tres percepciones básicas acerca de la motivación del ecosistema:

	\begin{enumerate}
		\item Los incentivos son la base que asegura los beneficios universales. La distribución incorrecta o despareja de estos puede llevar a una situación tal en la que el \textit{dinero malo} reemplace al \textit{dinero bueno}.
		\item Los incentivos deben ser continuos; los incentivos de corta duración pueden causar resultados negativos irreversibles.
		\item La escala de los incentivos debe ser apropiada.
	\end{enumerate}

	Nebulas considera siempre los incentivos como una parte esencial en el diseño de las características técnicas de su economía; se espera que estos incentivos beneficien a los miembros de la comunidad de manera más equitativa y aumenten significativamente su participación.

	\item

	\textbf{Colaboración inclusiva y eficiente.}

	Como Nebulas es una metanet autónoma legítima, posee la capacidad de lograr su auto-evolución sin necesidad de recurrir a \textit{hard-forks}. Dentro de su comunidad, una vez que la propuesta es aprobada por medio del voto \onchain, ya es posible realizar una actualización e implementarla de inmediato. Si llegara a surgir un problema en la red, es posible lanzar mejoras disponibles de forma inmediata. Los problemas futuros de Nebulas no serán como aquellos que existen en otros blockchains públicos como el de Ethereum, que está limitado por sus tecnologías y estrategias inmaduras.

	En paralelo a su eficiencia técnica, la gobernanza de Nebulas ofrece además un proceso transparente y sencillo (\ref{governance}) para mejorar la eficiencia de la colaboración.

\end{enumerate}