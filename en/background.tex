\section{Background}

The goal of Nebulas governance is to achieve the Nebulas vision and its focuses on decentralized collaboration. Before introducing the details of Nebulas governance, first you need to know the difficulty of collaboration via blockchain and the issues that Nebulas wants to solve.

\label{background}

Human beings are social and we are not strangers to Collaboration. Even Robinson on the island had a set of models to collaborate with Friday~\cite{robinson}. Collaboration itself does not have any absolute advantages or disadvantages. In different situations, one or more than one collaborations methods suitable for the situations should be selectd. With the developmeent of science and technology, the situation has been developed from face-to-face cooperation to a global collaboration accross the locations and organizations. The goal of collaboration is also more variable, the results raanging from physical to virtual, and the period becomes longer and more flexible.

Nebulas doesn't want to subvert other forms of collaboration, nor does it exclude multiple ways of collaborating. Nebulas tries to find a more appropriate way to collaborate to complement others in new situations. The new situations has the following features:

\begin{itemize}
	\item \textbf{Information interaction is developing from simple to complex.}

	The birth of the decentralized digital cryptocurrency, Bitcoin, allows the blockchain to records transaction information. The second-generation blockchain system, such as Ethereum, proposes a smart contract with Turing completeness, and the blockchain becomes programmable. Then there are more and more data and assets iinteraction problems in different situations such as on-chain, off-chain, and cross-chain.

	\item \textbf{The user roles are also increasing.}

	In the early Bitcoin community, there were only miners and token-holders. After Ethereum, there were developers, users. And more and more people know blockchain, then it becomes a challenge to distribute of rights and responsibilities of different user roles.

\end{itemize}

The problems such as:

\begin{enumerate}
	\item 

	\textbf{Centralized governance can't face to the complex new situation.}

	Blockchain is essentially a decentralized, untrusted, game-based autonomous system. Its true charm is the open collaboration model based on the consensus mechanism and the decentralization idea~\cite{whitepaper}. However, some blockchain projects simply do the opposite and use centralization way to govern, such as directly arbitraating hackers through the core arbitration court. The legitimacy and fairness of this approach are difficult to guarantee. Faced with complex data interaction patterns and rich user roles, the centralization of single evaluation criteria is difficult to achieve a wide and comprehensive, causing community members to resist. On January 11, 2019, the EOS Authority initiated a vote on whether to delete the ECAF (The EOSIO Core Arbitration Forum), and the percentage of support was as high as 98\%~\cite{DeleteECAF}。

	\item 

	\textbf{The existing decentralized governance rules are not uniform.}

	For example, in the Bitcoin community, for users of different roles such as miners and holders, the corresponding rules are not the same and are not clear. Such decentralized governance methods are likely to cause unclear community development goals, and it is difficult to effectively organize and iteratively upgrade.

	\item 

	\textbf{Traditional decentralized collaboration is often in the tragedy of the commons~\cite{TragedyOfTheCommons}.}

	Traditional decentralized collaborative projects, such as a large number of open source communities, have unclear interest models. Funding sources are often based on donations. Upgrades and evolutions rely too much on developers' interests. Frequently, there are problems such as the tragedy of the commons and the slow evolution of ecology. There are more people using public resources (such as open source code) and fewer people contribute. Open source projects that rely on large companies and enterprises to donate donations are often blocked by the development direction of large companies and become affiliates of enterprises.

	Blockchain technology, because of the existence of tokens, gives us the opportunity to solve the basic dilemma of decentralized collaboration by providing sustainable incentives and building a benign economy.

	\item

	\textbf{The early blockchain project consensus mechanism incentives are not comprehensive, and community participation is low. }

	For example, the Proof of Work (PoW~\cite{pow}) used by Bitcoin only focuses on mining incentives. This single incentive cannot cope with the gradual enrichment of user roles. Decentralized Ethereum has been criticized for its slow upgrades. Ethereum's upgrade proposal needs to be widely recognized by the community and then executed by the miners. However, the opinions of the parties in the entire ecology are difficult to unify. The incentives of Ethereum do not cover the different user roles of the entire ecosystem. The phenomenon of “doing nothing” has led to low participation in the Ethereum upgrade proposal, delay in implementation, and impediment to ecological development.

\end{enumerate}

There is currently no perfect solution to solve the above problems. We realize that in the new world with unprecedented complexity, the birth of new technologies is expected.

