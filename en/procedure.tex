\section{On-chain Vote and Incentive}

星云聚焦链上治理,致力于运用区块链技术来提供更公平的协作环境。

\subsection{链上治理流程}
\label{governance}

星云链上治理通用流程如下:

\begin{enumerate}
	\item \textbf{Proposal Period}: 发起人在社区公开发起提案,在提案投票期间,如果通过了NAT链上投票,则项目立项;
	\item \textbf{Develop Period}: 项目立项,提案人本人或经本人认可的社区成员按计划执行;
	\item \textbf{Testing Period}: 执行人提交结果,在公测投票期间,如果通过了NAT链上投票,则进入下一阶段;
	\item \textbf{NBRE Period}: 通过了两次投票的项目,社区没有异议且经技术委员会验收核实,即可最终执行发布。
\end{enumerate}

其中,“NAT链上投票”是核心手段,包括两部分:

\begin{enumerate}
	\item 投票唯一介质:NAT及其底层算法是星云链上治理的重要工具;
	\item 链上投票流程。
\end{enumerate}

本章节将对其进行重点介绍。

\subsection{投票基本原则}

星云生态通过星云主网实现链上投票。社区投出的每一票,将在星云链的区块上公开透明地展现。在星云链的系统中,投票将遵循以下基本原则:

\begin{enumerate}
	\item 投票的最基本单位是一个星云主网地址。
	\item 星云链的投票权重将参照地址的星云指数。
	\item 对系统有积极贡献的行为应被奖励更多的投票权,在投票的场景中,我们认为投票行为是对星云系统有积极贡献的行为,应被激励更多的投票权。
	\item 投票唯一介质NAT是星云指数的资产化表现。
\end{enumerate}

\subsection{投票方式}

投票将通过星云主网上的投票智能合约实现,每个地址可以选择投赞同、反对、弃权三种票。亦可不参与投票。

\subsection{The only voting medium: NAT}
\label{nat}

\subsubsection{Overview}

\begin{itemize}
	\item \textbf{Name}: Nebulas Autonomous Token
	\item \textbf{Symbol}: NAT
	\item \textbf{Form}: NRC20 token
\end{itemize}

The Nebulas Autonomous Token (NAT) is the asset derived from Nebulas Rank which will be embodied in the form of a NRC-20 Token and will serve as the only voting medium within the Nebulas governance ecosystem.

\begin{quote}
	\textbf{What's Nebulas Rank?}
		
	Nebulas Rank (NR) is the first on-chain, native, multidimensional value measurement mechanism for blockchain data.

	Within the Nebulas economy, the basic unit of governance is an “address” (\ref{rights}). Nebulas Rank quantifies the contribution of each “individual” to the economic accumulation via mathematical expression of the contribution to each address. Nebulas Rank is divided into “Core Nebulas Rank” and “Extended Nebulas Rank”. “Core Nebulas Rank” primarily refers to two factors:

	\begin{enumerate}
		\item The median value of the account within a certain period of time.
		\item The degree of access to the account over a certain period of time.
	\end{enumerate}

	At the macro level, the relationship between the number of currencies, the value of money, the rate of circulation, and productivity in the blockchain is described by the classical equation of the quantity of money in economics. The Nebulas Rank of the entire network can reflect the overall liquidity of the Nebulas ecosystem and activity.

	\vspace{2em}

	\textbf{NAT and NR}

	NAT的发行主要参考“核心星云指数”,是核心星云指数的资产表现,NAT发行将以周为单位,参考一周之内地址的资产中值和出入度衡量计算所得的星云指数。关于星云指数的更多信息,请参考2018年6月星云研究院发布的《星云指数黄皮书》。

	\vspace{2em}

	\textbf{How to check your Nebulas Rank?}

	Nebulas Rank was on-chain when Nebulas NOVA~\cite{nova} finished the first voting on May 6, 2019, it can be updated through the core ability of Nebulas NOVA, NBRE (Nebulas Blockchain Runtime Environment). The core Nebulas Rank is open source, you can check it online~\cite{CheckNR}.

\end{quote}

\subsubsection{Use cases}

NAT是星云治理中,链上投票场景的唯一介质,支持星云推进社区的治理。社区成员可以通过NAT进行链上投票表达自己对星云生态的意见,包括但不限于星云理事会的选举、星云主网通过星云区块链可执行环境(NBRE)执行星云协议表示(Nebulas Protocol Representation,NRR)、社区提案的立项、项目成果公测等。

\subsubsection{发行}
	
NAT的发行方式和比特币类似,在总量存在上限的前提下,按一定周期内全网星云指数的情况,以周为单位递减释放。

NAT协议的数量上限和星云链主网的全网星云指数值相关,释放量按周递减。递减系数为:$\lambda$。初始数值$\lambda$=0.997,即约在第180周时,发行量递减为第一周的58\%。

NAT的初始发行量以2019年5月6日星云主网Nebulas NOVA完成第一次投票升级后的全网星云指数为参照。Based on the current Nebulas Rank of the entire network and the current initial parameters, the upper limit of the total amount of NAT to ever exist will be 100 billion.

\subsubsection{Manage NAT}

Users can manage their NAT via NAS nano Pro~\cite{NASnano} and other apps which support NRC20~\cite{wallets}. At the same time, users can check the transactions and 流通情况 on the block explorer~\cite{explorer} which supports Nebulas mainnet.

\subsection{Get NAT}

All users who own Nebulas mainnet addresses have chances to get NAT. Except the addresses on the black list, users own Nebulas mainnet addresses can get NAT via three ways: improve NR score of the address, take part in the Nebulas on-chain voting, and pleding NAS.

\begin{quotation}

\textbf{NAT的黑名单地址}

在NAT的发行过程中,与星云“地址”基本权利主张(参见~\ref{rights})任何一条产生冲突的地址将被归为黑名单地址。黑名单地址只能根据其享有的权利获得部分NAT。

例如中心化交易所的地址就被归为黑名单地址。依照星云地址第一点基本权利主张,该地址具备拥有和操作星云链上资产的权利,所以交易所的归集地址可以在同等条件下按照地址的星云指数获得NAT,但这部分NAT的产权应属于对应的交易所用户。依照星云地址第二、第三点基本权利主张,在交易所证明该归集地址充分代表了相应托管资产用户提案和投票意愿之前,交易所归集地址并不具备发起提案和参与提案投票的权利,因此亦不能通过参与投票获得投票部分的NAT激励。

\end{quotation}

\subsubsection{Receive NAT through improving NR score of the address}

NAT tokens will be sent to the Nebulas mainnet address which has NR score on a weekly basis. 发放的数量将参照该地址上一周的星云指数和星云全网星云指数的情况。

The number of tokens per week will be decremented. The decreasing coefficient is $\lambda$. Initially $\lambda$ = 0.997.

In the $i$th week, the raatio is:

\begin{align}
1 \text{NR}=z(x_{ne},x_{e},\mu)\times\lambda^{i} \text{NAT}
\end{align} 

Above Formula breakdown:

\begin{itemize}
	\item $\lambda$: attenuation coefficient.
	\item $\mu$: incentive parameters for voting behavior.
	\item $x_{ne}$: the sum of the NR of non-exchange address of the entire network.
	\item $x_{e}$: NR sum of the entire website exchange address.
	\item $z(x_{ne},x_{e},\mu)$:以$x_{ne}$、$x_{e}$和$\mu$为变量的函数,星云指数和NAT的兑换比例。
\end{itemize}

\subsubsection{通过质押星云链主网原生代币(NAS)获得NAT}

从2019年5月6日开始,星云主网地址用户可以选择向投票的智能合约质押星云主网原生代币星云币(NAS)获得NAT。

质押NAS的用户将从质押开始后的第2周(即不早于2019年5月13日)开始获得NAT的空投。如果用户取回质押的NAS,则停止获得空投。

质押NAS的用户每周获得空投的数量递减,递减系数为$\lambda$,初始数值$\lambda$=0.997,

第$i$周质押NAS获得NAT的比例为: 

\begin{align}
x \text{NAS} \rightarrow \alpha \times z(x_{ne},x_{e},\mu)\times g(x) \times \lambda^{i} \text{NAT}
\end{align}

其中:

\begin{itemize}
	\item $x$:质押NAS的数量;
	\item $\alpha$:质押系数,初始数值$\alpha$=5;
	\item $z(x_{ne},x_{e},\mu)$:以$x_{ne}$、$x_{e}$和$\mu$为变量的函数,星云指数和NAT的兑换比例;
	\item $g(x)$:与$x$相关的函数,用于模拟以$x$值的NAS在星云主网获得的星云指数的情况。
\end{itemize}

\vspace{2em}

\textbf{如何发起质押?} 
	
用户可以通过使用星云钱包NAS nano Pro或其他支持NAS的客户端向投票智能合约发送交易,确认要质押NAS的数量。

为了保证用户可以获得并管理NAT,用户需要用自己掌握私钥的主网地址向星云的投票智能合约发送NAS完成质押,请勿使用交易所账户发送交易。

\vspace{2em}

\textbf{如何取消质押?}

用户可以通过NAS nano Pro或其他客户端调用智能合约申请取消,取消后可立即取回质押的NAS,取消质押后,则停止获得NAT空投。

\subsubsection{Receive NAT through Nebulas on-chain voting}

NAT will be conducted at the beginning of every week on the Nebulas mainnet. Once addresses obtain NAT, they can choose to vote on various proposals and elections. Available voting options are “Yes,” “No,” or “Abstain” and each choice is a valid option to receive incentives. If a user does not participate in any votes during the weekly airdrop cycle, they will not receive any additional incentives the following week.

\vspace{2em}

\textbf{Proportion of incentives}

The distribution and proportion of incentives should be fair and not used maliciously. To assist with these standards, the weekly NAT airdrop will look at the following:

\begin{enumerate}
	\item The number of NAT the address utilized for voting during the week.
	\item The amount of NAT tokens to be received this week based on the address’ NR score from the previous week.
\end{enumerate}

If a person utilizes their NAT for voting, they qualify to receive a NAT reward. However, if the same address would receive the NAT airdrop due to their NR score, they would not receive both the airdrop and the additional reward. Instead, they will receive the lower amount of the two.
	
During the $i$th week, NAT incentive distribution on the maninet address, the following formula will be used:

\begin{align}
\mu\times \min\{N_{v},N_{nr}\}
\end{align}

Above Formula breakdown:

\begin{itemize}
	\item $\mu$: the incentive parameters, $\mu$=10 under the initial parameters.
	\item $N_{v}$: the amount of NAT that was sent out from the address for the week.
	\item $N_{nr}$: how much NAT the address will receive this week based on the previous weeks Nebulas Rank score.
\end{itemize}

When $N_{v}$ (sent by the address in the week) is less than or equal to $N_{nr}$, the number of incentive NAT obtained will be $\mu\times Nv$. When the $N_{v}$ of the address is greater than $N_{nr}$, the amount incentive obtained will be $\mu\times N_{nr}$.

For example:

An address obtains 10 NAT based on its NR score from the previous week and there is a total of 1,000 NAT held within the address.

This week, the address votes 5 NAT that is less than 10 NAT based on its NR score from the previous week, and in return, will receive $10\times5=50$ NAT voting incentive.

If the address votes 1,000 NAT, more than 10 NAT, and in return, will receive 10\times10=100$ NAT voting incentive.

Similar to the weekly NAT and the NAS pledge program, the distribution of the NAT voting incentive is also decremented weekly by the same coefficient. Under the initial parameters, $\lambda$, the coefficient is $\lambda$=0.997.

\subsection{投票规则}
	
\subsubsection{投票手续费}

每此投票将收取$\theta$\% NAT将作为投票手续费,此部分手续费由星云理事会授权交由星云基金会作为NAT项目的专项运营资金管理,项目团队不得将此部分手续费直接用于投票。初始数值$\theta$=3。

\subsubsection{投票和NAT销毁}

在每个发行周期内,用户投入星云链投票智能合约的NAT将被立即销毁,销毁的比例会按照周期递减,递减速率和NAT发行的递减速率一致。每个周期内销毁部分的NAT将按照NAT销毁速率函数计算,具体公式参见附录\ref{burn}。

\subsubsection{投票通过标准}
	
投票是否通过将通过两个维度的标准来衡量:投票的参与度和赞成票的占比。

\begin{enumerate}
	\item 

	\textbf{投票参与度:}

	对于涉及到使用公共资产支持的提案,投票的参与度不得低于该提案提起资产占全网流通资产的比例。

	如某提案要求动用X NAS支持,此时星云主网中流通的NAS(任何未在锁仓/质押状态、可随时在星云主网上进行转账交易的NAS)为Y。

	则此提案通过需要达成的全网投票参与度不得低于X/Y,换算成NAT来表示,及参与此次投票的NAT与该周期初期空投给用户的NAT的比例不得低于X/Y。

	对于不涉及到使用公共资产支持的提案,投票的参与度由社区共同决定,此类提案包括但不限于星云主网参数的调整、NBRE要执行的NPR等。

	\item

	\textbf{赞成票的占比:}

	在满足投票最低参与度之外,某一提案投票是否通过还需要满足赞成票占总投入票数的比例不得低于51\%。

	即假设某一提案共收到票数为N,其中赞成票为Y,反对票为N,弃权票为A,则只有当Y/(Y+N+A) >= 51\%时,此提案才被视为投票通过。
\end{enumerate}

\subsection{投票监督和管理}

\subsubsection{投票流程监督}
\label{second-vote}

星云技术委员会受星云理事会委任,负责监督治理流程,保证整个流程公开透明。星云社区链上公开投票由星云技术委员会负责组织和管理。

公开投票接受社区所有成员的公开监督。针对违反星云基本权利主张的提案,星云技术委员会可以向星云理事会发起重审提案申请。星云理事会作为星云生态中治理流程正当性的监督者,有权对某一提案提起且仅能提起一次进行“\textbf{二次投票}”的要求。

当理事会提出“二次投票”的要求时,该提案被视为进入到新的投票周期进行一次新的投票。第一次投票过程中的结果不被执行,第一次投票投出的NAT不予返还,会按照当周期的烧毁速率进行烧毁。

二次投票的投票参与度需大于第一次投票的参与度。即假设第一次投票的参与度为X/Y,则第二次投票的参与度应大于X/Y,且赞成票的比例不低于51\%,方可视为投票通过。

\subsubsection{NAT参数调整}

NAT的发行过程涉及到如下系数:

\begin{enumerate}
	\item $\alpha$:质押系数,初始数值$\alpha$=5
	\item $\mu$:投票奖励系数,初始数值$\mu$=10
	\item $\lambda$:递减系数,初始数值$\lambda$=0.997
	\item $\theta$:投票手续费,初始数值$\theta$=3
\end{enumerate}

系数的调整需要经过星云生态的治理投票流程,星云基金会或NAT项目团队无权擅自调整系数。




